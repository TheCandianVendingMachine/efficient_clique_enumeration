\documentclass{article}
\usepackage{amsmath,amsthm,amssymb}
\usepackage{color}
\usepackage{listings}
\usepackage{caption}

\theoremstyle{plain}
\newtheorem{thm}{Theorem}[section]
\newtheorem{lem}[thm]{Lemma}
\newtheorem{prop}[thm]{Proposition}
\newtheorem{cor}[thm]{Corollary}

\theoremstyle{definition}
\newtheorem{defn}{Definition}[]

\theoremstyle{remark}
\newtheorem*{rem}{Remark}
\newtheorem*{note}{Note}
\newtheorem*{case}{Case}

% taken from https://tex.stackexchange.com/a/218450
\newcounter{nalg}[section] % defines algorithm counter for chapter-level
\renewcommand{\thenalg}{\thesection .\arabic{nalg}} %defines appearance of the algorithm counter
\DeclareCaptionLabelFormat{algocaption}{Algorithm \thenalg} % defines a new caption label as Algorithm x.y

\lstnewenvironment{algorithm}[1][] %defines the algorithm listing environment
{
    \refstepcounter{nalg} %increments algorithm number
    \captionsetup{labelformat=algocaption,labelsep=colon} %defines the caption setup for: it ises label format as the declared caption label above and makes label and caption text to be separated by a ':'
    \lstset{ %this is the stype
        mathescape=true,
        frame=tB,
        numbers=left,
        numberstyle=\tiny,
        basicstyle=\scriptsize,
        keywordstyle=\color{black}\bfseries\em,
        keywords={,input, output, return, datatype, function, in, if, else, foreach, while, begin, end, } %add the keywords you want, or load a language as Rubens explains in his comment above.
        numbers=left,
        xleftmargin=.04\textwidth,
        #1 % this is to add specific settings to an usage of this environment (for instnce, the caption and referable label)
    }
}
{}

\DeclareMathAlphabet{\mathcal}{OMS}{cmsy}{m}{n}
\SetMathAlphabet{\mathcal}{bold}{OMS}{cmsy}{b}{n}

\newcommand{\bigO}{\mathcal{O}}
\newcommand{\cn}[1]{ {{CN}_{#1}} }
\newcommand{\ce}[1]{ {{CE}_{#1}} }
\newcommand{\vdeg}[1]{ {\text{deg}({#1})} }
\newcommand{\ignv}[1]{ {I_{#1}} }
\newcommand{\nev}[1]{ {N_{#1}} }
\newcommand{\mdeg}{ {\Delta(G)} }

\title{An Efficient Algorithm for Listing Maximal Cliques in Arbitrary Graphs}
\author{Bailey Danyluk}
\date{April 21, 2024}


\begin{document}
\maketitle

\begin{abstract}
The clique problem has previously been known to be solvable in exponential time (!! CITE BEST KNOWN CASE !!). In this
paper, I outline new properties of maximal cliques such that they are able to be
enumerated in a worst case of \(\bigO(|V|^6)\).
\end{abstract}

\section{Introduction}
TBD

\section{New properties of a clique in a given graph}

In this section, a series of sets will be described that can be constructed such that they
can identify cliques with only set operations on a vertex and it's neighbours. Likewise,
the maximum size of every relevant set will be proved for algorithm analysis in a later
section.

\subsection{Preliminary definitions}
% Define what a clique is
\begin{defn}
    Let \(G\) be some graph with vertices \(V\). The set \(\{ C_1, C_2, \dots, C_k \}\)
    where \(C_i \in V\) is a member of an arbitrary maximal \(k\)-clique in \(G\). We call
    this set \(C\).
    \label{clique_def}
\end{defn}

% Define the vertices which are adjacent which are not apart of the clique
\begin{defn}
    Let \(u \in C\). \(\ignv{u}\) is the set of neighbours of \(u\) which are not in \(C\).
    That is, \(\ignv{u} = \{ \ignv{u}_1, \ignv{u}_2, \dots, \ignv{u}_i \}\) where
    \(\ignv{u}_j\) is a neighbour of \(u\) such that \(\ignv{u} \cap C = \emptyset\).
\end{defn}

\begin{rem}
    The assumption that \(C\) is maximal can be made since \(C\) is not constructed; it is
    just a subset that must exist within the neighbours of some \(u \in V\). If \(C\) is
    not maximal, we can remove the missing vertices from \(\ignv{u}\) and add them to
    \(C\).
\end{rem}

% Define the neighbours of a vertex
\begin{defn}
    Let \(u \in C\). \(\nev{u}\) are all vertices reachable in at most a single step from
    \(u\). That is, \(\nev{u} = \ignv{u} \cup C\).
    \begin{note}
        This is different than just all vertices adjacent to \(u\). \(u \in \nev{u}\)
        since \(u \in C\). This follows from the definition since \(u\) is zero steps from
        itself.
    \end{note}
\end{defn}

\begin{cor}
    Let \(u \in V\). \(|\nev{u}| = \vdeg{u} + 1\).
\end{cor}

\begin{cor}
    \(\forall u \in V\), \(|\nev{u}| \leq |V|\). \label{neighbour_size}
\end{cor}

\begin{rem}
    Every vertex in a graph is apart of some clique. This clique may just be \(K_1\) or
    \(K_2\), but these are maximal cliques which need to be counted to properly create the
    list.
\end{rem}

\subsection{Common Neighbours of a Vertex}

% Define common neighbours
\begin{defn}
    Let \(u \in V\). \(\cn{u}_i\) is the set of vertices in \(\nev{u}\) which \(u\) shares
    with it's \(i\)-th neighbour \(\nev{u}_i\).
\end{defn}

\begin{defn}
    \(\cn{u}\) is the set which lists all neighbours which \(u\) has in common with any of
    it's neighbours. That is, if \(u\) can reach a vertex \(v\) and some neighbour of
    \(u\) can also reach the vertex \(v\), then \(v\) exists within some subset of
    \(\cn{u}\).
\end{defn}

We can construct \(\cn{u}\) by intersecting \(\nev{u}\) with the neighbours of \(u\)'s
i-th neighbour.
\begin{equation}
    \cn{u} = \bigcup_{i=1}^{\vdeg{u}} \{ \nev{u} \cap \nev{v} \}, v = {\nev{u}}_i
    \label{cn_construct}
\end{equation}

\begin{lem}
    Let \(u, v \in C\) where \(u \neq v\) and \(\ignv{u} \cap \ignv{v} = \emptyset\). Then
    \(\cn{u}_i \cap \cn{v}_j = C\) for any \(i, j\).
    \label{cn_no_common}
    \begin{proof}
        Let \(u, v \in C\) such that \(u \neq v\), and \(i,j\) be some index of \(\cn{u},
        \cn{v}\). \(\cn{u}_i \cap \cn{v}_j = (\ignv{u} \cup C) \cap (\ignv{v} \cup C)\).
        Since \(\ignv{u} \cap \ignv{v} = \emptyset\), we can see that the previous
        statement is equal to \(C = C \cup (\ignv{u} \cap \ignv{v}) = C \cup \emptyset\).
    \end{proof}
\end{lem}

\begin{cor}
    For some \(u \in V\), \(\cn{u}\) only contains vertices which \(u\) is adjacent to.
    \label{cn_exclusive_neighbours}
\end{cor}

% Prove size of Common Neighbours
\begin{cor}
    \(\forall u \in V\), \(|\cn{u}| = \vdeg{u}\)
\end{cor}
\begin{cor}
    \(\forall u \in V\), \(|\cn{u}| \leq |V| - 1\)
    \label{max_cn}
\end{cor}

So, \(\cn{u}\) is a set of neighbours which \(u\) has in common with an adjacent vertex
\(v\). This set will be used in the next section to create another set which will list all
edges which \(u\) shares with all of it's neighbours.

\subsection{Common Edges of a Vertex}

% Define Common Edges
\begin{defn}
    Let \(u \in V\). \(\ce{u}_i\) is the set of edges in \(\nev{u}\) which \(u\) shares
    with it's \(i\)-th neighbour \({\nev{u}}_i\).
\end{defn}

We can construct \(\ce{u}_i\) by intersecting \(\cn{u}_i\) with the j-th neighbour's
common neighbours, \(\forall \cn{(\nev{u}_j)}\). When we take the union of all of these
intersections, we will have constructed \(\ce{u}\).

% Prove that these are indeed common edges
\begin{equation}
    e(u, n, i, j) =
    \begin{cases}
        \cn{u}_i \cap {\cn{n}}_j,n \in \nev{u},&
            \text{if } u \in (\cn{u} \cap \cn{n}) \\
            & \text{ and } n \in (\cn{u} \cap \cn{n})
        \\
        \emptyset,& \text{otherwise}
    \end{cases}
    \label{edge_construction}
\end{equation}

\begin{rem}
    We do have to check if both vertices are in the resulting intersection. Consider a
    path graph, and select a vertex, \(u\), not on the ends. Both neighbours of this path
    graph will attempt to intersect their neighbours \(\neq u\) with \(u\). The resulting
    set will be the set that only contains the neighbour. This doesn't fit our definition
    of \(CE{u}_i\), so we ignore these values.
\end{rem}

\begin{lem}
    Equation (\ref{edge_construction}) will be a set that strictly contains vertices that
    both \(u \in V\) and it's \(k\)-th neighbour have an edge to.
    \begin{proof}
        Let \(u \in V\), \(E_u = \cn{u}_i\) and \(E_v = \cn{v}_j\) where \(v = \nev{u}_k\)
        for some \(i\), \(j\). Because \(u\) and \(v\) are neighbours, an edge exists
        between \(u\) and all vertices in \(E_u\) from (\ref{cn_exclusive_neighbours}).
        Likewise, the same argument applies for \(v\) and \(E_v\). When we intersect
        \(E_u \cap E_v\), the resulting set will be vertices that both \(u\) and \(v\)
        have an edge to.
    \end{proof}
    \label{unique_edges}
\end{lem}

% Define CE
\begin{defn}
    Let \(u \in V\). \(\ce{u}\) is the set which contains sets of all edges,
    \(e(u, n, i ,j)\), which \(u\) has in common with all of it's neighbours
    \(n \in \nev{u}\).
\end{defn}

% Construct the entire common edge set
\begin{lem}
    For some \(u \in V\)
    \begin{equation}
        \ce{u} = \bigcup_{i=1}^{|\cn{u}|}
            \bigcup_{k=1}^{|\nev{u}|}
                \{ n = \nev{u}_k \mid
                \bigcup_{j=1}^{|\cn{n}|} \{ e(u, n, i, j) \} \}
    \end{equation}
    \begin{proof}
        Let \(u \in V\). What this abuse of notation does is: for the \(i\)-th iteration,
        we take the \(i\)-th common neighbour of \(u\) and test it against every neighbours
        common neighbours using the function we defined in (\ref{edge_construction}). So,
        we will get a set of common edges that both \(u\) and every neighbour in
        \(\nev{u}\) have. This is the definition of \(\ce{u}\).
    \end{proof}
    \label{ce_construct}
\end{lem}

% Prove that the set is bounded by polynomial size
\begin{lem}
    Let \(u \in V\). \(|\ce{u}| \leq |V|^3 \)
    \begin{proof}
        We know that for any \(u \in V\), \(|\cn{u}| = |V|\) from (\ref{max_cn}). Likewise,
        the maximum size of \(\nev{u} = |V|\) as well from (\ref{neighbour_size}). So,
        (\ref{ce_construct}) will take at most \(|V| \cdot |V| \cdot |V| = |V|^3\)
        intersections to complete, only ever constructing \(|V|^3\) sets that all get
        unioned into the main \(\ce{u}\) set.

        \(\therefore |\ce{u}| \leq |V|^3\).
    \end{proof}
    \label{max_ce}
\end{lem}

% Prove C in CE theorem
\begin{thm}
    \(u \in C \iff C \in \ce{u}\).

    \begin{proof}[Proof of \(u \in C \implies C \in \ce{u}\)]
        Let \(u, v \in C\) where \(u \neq v\).
        \begin{note}
            \(C \in \nev{u}\) and \(C \in \nev{v}\) since \(u\) and \(v\) are chosen from
            the same clique.
        \end{note}
        There are two cases to consider:
        \begin{case}[\(\ignv{u}\) has no elements in common with \(\ignv{v}\)]
            So, \(\ignv{u} \cap \ignv{v} = \emptyset\).

            From (\ref{cn_no_common})
            we know that \(C \in \cn{u}_i\) and \(C \in \cn{v}_j\). When we construct
            \(\ce{u}\), we will intersect \(\cn{u}_i \cap \cn{v}_j = C\).

            \(\therefore C \in \ce{u}\).
        \end{case}
        \begin{case}[\(\ignv{u}\) has at least one element in common with \(\ignv{v}\)]
            not true
        \end{case}
    \end{proof}

    \begin{proof}[Proof of \(u \in C \impliedby C \in \ce{u}\)]
        Let \(u \in V\). Assume \(C \in \ce{u}\), but \(u \not\in C\). Pick \(v \in C\).

        \begin{case}[\(v \in \nev{u}\)]
            We will pick another \(v \in C\) such that \(v \not\in \nev{u}\). If we cannot
            pick such a \(v\), that means that \(C\) is not maximal, or \(u \in C\).
        \end{case}
        \begin{case}[\(v \not\in \nev{u}\)]
            From (\ref{cn_exclusive_neighbours}),
            \(\forall i \leq \cn{u}, v \not\in \cn{u}_i\). However, for \(C \in \ce{u}\),
            we must intersect \(\cn{u}\) with one of it's neighbours \(\cn{\nev{u}}\) so
            that \(C\) is created. But since \(v\) is not in \(\cn{u}\) at all, we would
            be missing \(v\), so \((C \setminus v) \in \ce{u}\). This is a contradiction
            that \(C \in \ce{u}\).

            \(\therefore\) \(u\) must have an edge to \(v\). However, if we create this
            edge, then \(u \in C\).
        \end{case}
        From the two cases, we see if \(C \in \ce{u}\) then \(u \in C\).
    \end{proof}
    \label{c_in_ce}
\end{thm}

\(\ce{u}\) is the set which lists all edges which \(u\) has in common with all of it's
neighbours. That is, if \(u\) has an edge to vertex \(v\) and some neighbour of \(u\) also
has an edge to the vertex \(v\), then \(v\) exists within some subset of \(\ce{u}\). We
know that \(\ce{u}\)'s length is bounded by a polynomial (\ref{max_ce}), and that only
vertices in a clique \(C\) will contain \(C \in \ce{u}\). Using these facts, in the next
section we will develop an algorithm to list all cliques in a graph.

\section{An algorithm to list maximal cliques in an arbitrary graph}
In this section I will present an algorithm in multiple parts which, when combined, will
list all maximal cliques in an arbitrary graph.

\subsection{Algorithm}
\begin{algorithm}[caption={Create Common Neighbour Set}, label={algo_cn_construct}]
function: list_common_neighbours
    input: Set of vertices V, Set of tuples $(v_i, v_j)$ called E
    output: Set of common neighbours
    begin
        edge_reports $\gets$ {}
        for v in V
            neighbours $\gets$ adjacent vertices of v
            for n in neighbours
                edge $\gets$ the edge $(v, n) \in E$
                if edge in edge_reports
                    edge_reports[edge][2] $\gets$ neighbours
                else
                    edge_reports[edge] $\gets$ {neighbours,$\emptyset$}
                end
            end
        end

        common_neighbours = {}
        for edge in E
            ce $\gets$ edge_reports[edge][1] $\cap$ edge_reports[edge][2]
            add ce to set common_neighbours[edge[1]]
            add ce to set common_neighbours[edge[2]]
        end

        return common_neighbours
    end
\end{algorithm}

\begin{algorithm}[caption={Create Common Edge Set}, label={algo_ce_construct}]
function: list_common_edges
    input: Set of vertices V, Set of tuples $(v_i, v_j)$ called E
    output: Set of common edges
    begin
        common_neighbours $\gets$ common_neighbours(V, E)
        common_edges $\gets$ {}

        for u in V
            for v in neighbour vertices of u
                for Ucn in common_neighbours[u]
                    for Vcn in common_neighbours[v]
                        let ce = Ucn $\cap$ Vcn
                        if u in ce and v in ce
                            add ce to set common_edges[u]
                        end
                    end
                end
            end
        end

        return common_edges
    end
\end{algorithm}

\begin{algorithm}[caption={List Maximal Cliques}, label={algo_list_cliques}]
function: list_maximal_cliques
    input: Set of vertices V, Set of tuples $(v_i, v_j)$ called E
    output: Set of maximal cliques in the graph $(V, E)$
    begin
        common_edges $\gets$ common_edges(V, E)

        cliques $\gets$ {}
        for u in V
            for Uce in common_edges[u]
                clique $\gets$ {u}
                for n in neighbours of V
                    if Uce $\in$ common_edges[n]
                        add n to clique
                        remove Uce from common_edges[n]
                    end
                end

                add clique to cliques
                remove Uce from common_edges[n]
            end
        end

        // Remove sub-cliques so we are left with only maximal
        for c1 in cliques
            for c2 in cliques
                if c1 $\subset$ c2
                    remove c1 from cliques
                end
            end
        end

        return cliques
    end
\end{algorithm}

\begin{rem}
    The reason non-maximal cliques exist is that the properties listed in the previous
    section show that the maximal clique must exist, but are not necessarially
    exclusive, in \(\ce{}\). Using the algorithm as described, it will never misreport a
    clique since
\end{rem}

\section{Conclusion}

\begin{cor}
    \(P = NP\)
    \begin{proof}
        Proof is left as an exercise to the reader.
    \end{proof}
\end{cor}

\end{document}
